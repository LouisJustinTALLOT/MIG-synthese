\documentclass{article}

\usepackage[utf8]{inputenc}
\usepackage{graphicx}
\usepackage[bottom = 2 cm]{geometry}
\usepackage[french, english]{babel}
\usepackage{amsmath}
\usepackage{amsfonts}
\usepackage{amssymb}
\usepackage{numprint}
\usepackage{hyperref}
\usepackage[T1]{fontenc}
\usepackage{titling}
\usepackage[nottoc, numbib]{tocbibind}
\usepackage[dvipsnames]{xcolor}
\usepackage{fancyhdr}
\usepackage{blindtext}
\usepackage{titlesec}
\usepackage{titletoc}
\usepackage{array}
\usepackage{listings}
\usepackage{SIunits} %je crois qu'il est incompatible avec certains package (amssymb ?) mais il serait pourtant bien pratique. Une idée ? 
% je le trouve pas méga utile perso... exemples ?

\usepackage{biblatex}
\addbibresource{biblio.bib}
\bibliography{biblio.bib}

% \setlength{\hoffset}{-18pt}
% \setlength{\oddsidemargin}{0pt} % Marge gauche sur pages impaires
% \setlength{\evensidemargin}{9pt} % Marge gauche sur pages paires
% \setlength{\marginparwidth}{54pt} % Largeur de note dans la marge
% \setlength{\textwidth}{481pt} % Largeur de la zone de texte (17cm)
% \setlength{\voffset}{-18pt} % Bon pour DOS
% \setlength{\marginparsep}{7pt} % Séparation de la marge
% \setlength{\topmargin}{0pt} % Pas de marge en haut
\setlength{\headheight}{20pt} % Haut de page
% \setlength{\headsep}{10pt} % Entre le haut de page et le texte
\setlength{\footskip}{50pt} % Bas de page + séparation
\setlength{\textheight}{21cm} % Hauteur de la zone de texte (25cm)
\setlength{\parskip}{1em}
\setlength{\parindent}{2em}

\pagestyle{fancy}
\fancyhf{}
\rhead{École des Mines de Paris}
\lhead{MIG L'Eau après la Mine}
\rfoot{\vspace{-0.7 cm}\thepage}
\lfoot{\includegraphics[width = 3 cm]{logoMPT.png}}

%style de sections
\titleformat{\section}
    {\Large\bfseries}
    {\thesection}
    {1 em}
    {}
\titleformat{\subsection}
    {\large\bfseries}
    {\qquad\thesubsection}
    {1 em}
    {}
\titleformat{\subsubsection}
    {\large\bfseries}
    {\qquad\qquad\thesubsubsection}
    {1 em}
    {}

\renewcommand{\thesection}{\Roman{section})} % je préfére les parenthèses 
\renewcommand{\thesubsection}{\arabic{subsection})} % c'est + joli
\renewcommand{\thesubsubsection}{\alph{subsubsection})} 
% sinon on peut mettre Partie 1 ... aussi 

%paramétrage de la table des matières
\titlecontents{section}
            [0pt]
            {\bfshape}%
            {\contentsmargin{0pt}%
            \bfseries
            \makebox[0pt][r]{\large\thecontentslabel\enspace}%
            \large}
            {\contentsmargin{0pt}\large}
            {\hfill\textbf{\contentspage}}
            [\addvspace{1pc}]
\titlecontents{subsection}
            [20 pt]
            {}%
            {\contentsmargin{0pt}\bfseries            \makebox[0pt][r]{\large\thecontentslabel\enspace}%
            \large}
            {\contentsmargin{0pt}\large}
            {\hfill\textbf{\contentspage}}
            [\addvspace{1pc}]
            
\titlecontents{subsubsection}
            [40 pt]
            {}
            {\contentsmargin{0pt}\bfseries\makebox[0pt][r]{\large\thecontentslabel\enspace}\large}
            {\contentsmargin{0pt}\large}
            {\hfill\textbf{\contentspage}}
            [\addvspace{1pc}]
% Y a pas moyen de garder les petits points horizontaux (c'est plus lisible je pense) ?
% euh si si j'ai pas encore cherché comment c'est tout

\makeatletter
\renewcommand\listoffigures{%
    \section{\listfigurename}% Used to be \section*{\listfigurename}
      \@mkboth{\MakeUppercase\listfigurename}%
              {\MakeUppercase\listfigurename}%
    \@starttoc{lof}%
    }
\makeatother

\lstset{numbers=left, stepnumber=5, firstnumber=1, breaklines = true,backgroundcolor=\color{white}}%, numbersep=5pt}
\lstdefinelanguage{hytec}{language = Python,
    alsoletter =/,
    alsoletter =-,
    classoffset = 0,
    morekeywords = {darcy,velocity,diffusion-coeff,condition,coordinates,coeff,database,solver-regime,grid-regime,flow-regime,zone,head,porosity,permeability,geochem,boundary,exclude, unit,extend,kinetics,exclude,using,mineral,define,duration,timestep},
    keywordstyle={\color{blue}},
    classoffset = 1,
    % otherkeywords = {0,1,2,3,4,5,6,7,8,9,-,.},
    morekeywords = [2]{0,1,2,3,4,5,6,7,8,9,-,.},
    keywordstyle=[2]{\color{orange}},
    classoffset = 2,
    otherkeywords = {chimie_granite,chimie_granite_frac,chimie_residus_boues,chimie_residus_sable},
    morekeywords = [3]{chimie_granite,chimie_granite_frac,chimie_residus_boues,chimie_residus_sable,chimie_steriles,flux, top,gases, minerals,left,right,top},
    keywordstyle = [3]{\color{red}},
    classoffset = 4,
    morekeywords = [4]{tot, pH, fug,surface,power,rate,area,species,basis, variable,start,maximum,courant,factor,composition,logK,mg/l,umol/l,mmol/l,m/s,y-term,w-term},
    keywordstyle = [4]{\color{ForestGreen}}
}


\title{\textbf{\LARGE{\textsc{École des Mines de Paris}}\\ \vspace{1 cm}MIG L'Eau après la Mine\\\vspace{0.8 cm}Synthèse du projet}\vspace{1 cm}}

\author{Sophian \textsc{Akkari}, Tom \textsc{Boezennec}, Paul \textsc{Colombel}, Florestan \textsc{Fontaine},\\ Yiqiong \textsc{Hu}, Tasnime \textsc{Ouchtar}, Guillaume \textsc{Ramos}, Louison \textsc{Rapin}, Guillaume \textsc{Rouy},\\ Louis-Justin \textsc{Tallot}, Gabrielle \textsc{Vernet}, Guillaume \textsc{Vigne}, Robin \textsc{Willocquet}\\ \\ Irina \textsc{Sin}, Sophie \textsc{Guillon}, Nicolas \textsc{Seigneur}, Vincent \textsc{Lagneau}}

\date{\vspace{2 cm}Novembre - Décembre 2020}



\begin{document} % début du document
\selectlanguage{french}

\maketitle
\thispagestyle{empty}
\vspace{2 cm}
\begin{center}
    \includegraphics[width = 0.4\linewidth]{logoMPT.png}
\end{center}


\newpage
\pagenumbering{gobble}
\tableofcontents

\newpage

\begin{abstract}
    Ceci est le résumé
\end{abstract}


{\selectlanguage{english}
\begin{abstract}
    English abstract
\end{abstract}
}

\newpage
\pagenumbering{arabic}
\section*{Introduction}

\newpage

\section{Contexte de l’Après-Mine}
\subsection{Histoire minière de la France}

%Test pour voir la longueur (c'est notre texte mais on risque de l'améliorer, d'ajouter des images...) :+1:

%Ça donne quelque chose de très dense... (comme un latex en fait lol, mais un word (tu vois l'idée) est plus lisible car plus aéré. Y a moyen de rendre ça plus lisible ?

%coucou ! oui c'est très dense en effet je m'en occupe
% c'est bon normalement, du moins à mes yeux
% t'en pesnes quoi ? -LJ

\paragraph{} Lorsqu’on parle de mine, on a bien souvent en tête l’image du mineur de charbon portant casque avec lampe frontale ainsi que bleu de travail, noirci par son travail et éreinté de ses journées, qui descend chaque jour dans le ventre de la mine pour excaver la roche et nourrir sa famille. L’histoire minière française est bien plus diverse que cela en réalité : bien que le charbon occupe une place importante, surtout dans le Nord de la France, bien d’autres métaux ont été exploités en France - et notamment l’uranium.

Une mine est, en général, un site d’exploitation d’un minerai qui est nécessaire à diverses activités. 
La décision d’exploiter une mine est le résultat d’un processus appelé exploration. Il vise à établir, avec le plus haut niveau de confiance possible, la géologie du site. On effectue pour cela des sondages puis des forages et des analyses géochimiques. Si le filon est trouvé et que l’on ne sait rien de plus, on parle de ressource. Dès lors que l’on est assuré que l’exploitation est possible, on parle alors de réserve. Pour décider de l’exploitation, il faut réunir plusieurs conditions : une ressource en quantité suffisante, des techniques d’exploitation suffisamment adaptées et des risques financiers limités. Une fois l’exploitation décidée, vient la phase de développement. Cette dernière consiste à décider des techniques d’exploitation et à étudier les impacts sociaux et environnementaux afin d’obtenir les autorisations légales. Il se passe ainsi une dizaine d’années entre les premières études géologiques et le début de l’exploitation.
La moitié des mines sont à ciel ouvert, mais ces dernières représentent 90 \% de la quantité extraite. La plupart des mines existantes à ciel ouvert sont excavées par motif en escalier : cela permet de voir les différentes strates et différentes époques d’exploitation.

La France est un pays avec un passif minier bien présent et ancré dans les esprits. En effet, c’est vers l’ère industrielle qu’apparaissent de nombreuses mines qui mèneront à divers bassins miniers français : or, cuivre, fer, étain, manganèse, tungstène, bauxite…. Vers le début du XXème siècle, on compte environ 800 sites miniers en France. Les mines sont régulées par divers organismes, car les méthodes d’extraction peuvent être diverses et affecter de nombreuses sphères publiques.

L’Uranium, élément 92 du tableau périodique, est le centre de notre attention ici. En effet, il est au fondement de l’énergie nucléaire massivement utilisée en France et sert d’élément principal pour l’arme nucléaire. On en trouve naturellement, principalement de l’isotope 238. Et pourtant, c’est surtout l’isotope 235 qui nous intéresse : en effet, son activité radioactive est plus élevée. Il représente 0,7 \% de l’uranium mondial et comporte 2 degrés d’oxydation : le +IV se trouve surtout sous terre, il est insoluble et réducteur et est celui qui est prévalent (uraninite, coffinite) ; le +VI se trouve principalement en surface, et est soluble et oxydant (autunite). On le trouve aussi en quantité moindre en rétention dans de l’argile ou des oxydes.


En France, la volonté d’exploiter l’uranium date de la fin de la Seconde Guerre mondiale. Il s’agit alors d’un enjeu stratégique majeur puisque l’objectif était d’assurer l’approvisionnement en uranium de l’armée, qui en avait besoin pour développer l’arme nucléaire. La première mine d’uranium française a ainsi ouvert en 1948. Avec la décision de développer le nucléaire civil, il a ensuite fallu approvisionner en combustibles les centrales nucléaires pour assurer l’indépendance énergétique du pays. L’épuisement des gisements français et la découverte de gisements bien plus importants à l’étranger, notamment au Niger, entraînent la fermeture des mines françaises. La dernière mine d’uranium ferme ainsi en 2001. Au total, il y a près de 230 sites miniers liés à l’uranium en France, qui ont produit environ 76 000 t en un demi-siècle (une mine française produisait quelques tonnes d’uranium par an). Ces sites sont situés dans le Massif Central ainsi qu’en Bretagne et en Vendée.
Les gisements d’uranium ont des concentrations très diverses. La plupart des gisements ont une concentration en uranium de l’ordre de 0,1 \% mais elle peut atteindre jusqu’à 10 \% dans certaines mines canadiennes (Cigar Lake, McArthur River).

Nous avons eu l’occasion de visiter virtuellement deux mines d’uranium : la mine de Bellezane et celle de la Ribière. Le site de Bellezane comporte en fait deux mines à ciel ouvert. C’est un site en Haute-Vienne qui a été exploité de 1975 à 1992. 5 \% de la production d’uranium français provient du site de Bellezane, qui employait alors jusqu’à 100 personnes. Ce site, qui fait partie de la concession minière de la Gartempe, est un site historique français, de taille conséquente (25 kilomètres de galeries souterraines). De plus, il bénéficie de sa propre station de traitement des eaux, utilisée pour traiter environ 500000 $\meter^3$ d’eau par an.
Le site de la Ribière, abordé plus en détail dans ce rapport, est un site de plus petite taille, avec une teneur plus faible en uranium et un minéral extrait différent (de l’autunite, tandis qu’à Bellezane on extrait surtout de l’uraninite). Le site, dans la Creuse, avantagé par sa topographie, a été exploité de 1959 à 1984, puis a servi de lieu de traitement de l’uranium faible de 1982 à 1985. C’est, tout comme Bellezane, un site de stockage de résidus uranifères. Contrairement à Bellezane, il n’y a pas de station de traitement liée au site.


Les mines sont régies par différentes autorités : le code minier par exemple, qui pose un cadre juridique récent, ainsi que diverses polices des mines, dont la police des ICPE (installation classée pour la protection de l’environnement), créée en 1976. 
Des polices liées à des domaines transverses tels que l’eau, les déchets, ou encore la santé, interviennent elles aussi sur ce sujet qui lie ainsi de nombreux domaines. Le RGIE complète enfin ces polices des mines. (près IRSN ça)


Ces polices sont nécessaires car les risques miniers sont importants et peuvent causer de lourds dégâts. On peut parler d’instabilité mécanique, c’est à dire lorsqu’un pan de la mine s’effondre, s’affaisse, ou lorsque des digues créées pour retenir l’eau à la mine se rompent. Il y a aussi des risques d’impact sur l’hydrosystème : pollution de l’eau, inondation, ou sur le cycle de l’eau en lui même lorsque les eaux qui parcourent le site minier, aussi appelées eaux d’exhaure, sortent à la surface et viennent confluer dans d’autres cours d’eau voisins.
Après l’abandon d’une mine, c’est souvent ce qui arrive, et un ruisseau dit d’exhaure se forme alors. Enfin, les travaux miniers souterrains peuvent eux aussi affecter l’hydrosystème si l’eau se propage dans les galeries, voire créer un impact sur l’hydrodynamique même du milieu.

Puis, à partir de 1967, le nombre de mines tend à décroître. Ce sont donc les mines de charbon qui disparaissent en premier, puis celles de bauxite, d’argent… Mais bien qu’elles ne produisent plus, leur activité n’est pas finie pour autant. En effet, il s’agit de veiller à leur remise en état durable. C’est ce que l’on appelle l’\emph{après-mine}.

\subsection{Importance et enjeux de l’Après-Mine} 
\subsection{Enjeux sanitaires de l’Après-Mine}

\newpage
\section{Problématique de l’Après-Mine}
\subsection{Les différentes pollutions}
\subsection{La dimension sociétale}
\subsubsection{Focalisation du grand public sur l'uranium et les risques radiologiques}
\subsubsection{Gestion des risques par un acteur extérieur: rôle de l’IRSN}
\subsection{Les différentes normes mises en place sur l’eau}
\subsubsection{Acteurs de la législation} 
\subsubsection{Cas de l’eau rejetée dans l’environnement }
\subsubsection{Cas de l’eau potable} 
\subsubsection{Comparaison avec d’autres pays }

\newpage
\section{Traitements/Solutions}
\subsection{Différents traitements}
\subsubsection{Fonctionnement et coûts}
\paragraph{Traitement par résines échangeuses d’ions} 
\paragraph{Traitement par précipitations}
\paragraph{Traitement par zones humides}
\paragraph{Prélèvements et analyses}
\subsubsection{Comparaison, faisabilité}
\paragraph{Investissement et mise en place}
\paragraph{Exploitation et maintenance}
\paragraph{Gestion des déchets }
\subsubsection{Présentation des situations des autres pays}
\paragraph{Situation générale de mines d’uranium de chaque pays}
\paragraph{Les méthodes de traitement préférées par pays}

\subsection{Les traitements sont-ils toujours nécessaires ? Etude de cas du site de la Ribière}
\subsubsection{Présentation du site de La Ribière}
\subsubsection{Modèle géochimique et 1D}
\paragraph{Modélisation}
\paragraph{Résultats pour la chimie}
\subsubsection{Modèle hydrologique 2D}
\paragraph{Modélisation}
\paragraph{Résultats}
\paragraph{Fusion avec la géochimie - résultats}
\subsubsection{Conclusion pour le site}

\subsection{Le radon, un gaz volatile}

\newpage
\section{Synthèse des enjeux}
\subsection{Aspect socio-économique du traitement de l’eau}
\subsection{Ouverture à l’internationale}

\section*{Conclusion}

%%%%%% ANNEXES %%%%%%%
\newpage
\appendix
\pagenumbering{roman}
\titleformat{\section}
    {\Large\bfseries}
    {Annexe \thesection \: -}
    {0.5 em}
    {}
\newpage
\section{Références bibliographiques}
\printbibliography[header = none]

\section{Code HYTEC Utilisé }
% bonjour Irina

\lstinputlisting[language=hytec, caption = ziljbfjez]{modele-13.htc}
\newpage
\section{Résultats des simulations numériques HYTEC}


\newpage
\listoffigures


%La bibliographie se met avant les annexes normalement
% ouais okay je fais ça
% juste signez les commentaires c'est + cool -LJ

\end{document}
